% !TEX encoding = KOI8_R
\documentclass[oneside,final,14pt]{extreport}
\usepackage[koi8-r]{inputenc}
\usepackage[russianb]{babel}
\usepackage{amsmath}
\usepackage{vmargin}
\usepackage{graphicx}
\setpapersize{A4}
\setmarginsrb{1.5cm}{1.5cm}{1.5cm}{1.5cm}{0pt}{0mm}{0pt}{13mm}
\linespread{1.5}
%\setcounter{secnumdepth}{-1}
\usepackage{indentfirst}
\sloppy
\begin{document}
\begin{titlepage}
\begin{figure}[t!]
\centering
\includegraphics[width=0.7\linewidth]{msu}
\end{figure}
\begin{centering}
\textnf{Московский государственный университет имени М.В. Ломоносова}

\textnf{Факультет вычислительной математики и кибернетики}

\textnf{Кафедра алгоритмических языков}

\vskip1cm
\textnf{Отчёт о выполнении задания практикума}
\vskip1cm
\Large

\end{centering}
\vskip2cm
\normalsize
\begin{flushright}
\textnf{Выполнил:}

Студент 424 группы

\emph{Салимов Ярослав Даниилович}


\end{flushright}
\vfill
\centerline{\bf Москва ~ \bf 2019 \bf г.}
\end{titlepage}
\setcounter{page}{2}
\renewcommand{\contentsname}{Cодержание}
\tableofcontents

\chapter{Введение}
%\addcontentsline{toc}{chapter}{Введение}
В данной работе тестируется предположение, что цена некоторых рыночных активов подвержена влиянию настроений участников рынка. 
В качестве его измерения был выбран анализ тональности твитов, опубликованных в тематических аккаунтах.
Одной из актуальных задач современной науки является изучение поведения эукариотических клеток: их движение, изменение формы и движение отдельных частей клеток. Данная работа посвящена анализу движения внутриклеточных структур. В случае подвижных клеток вся система координат клетки движется относительно системы отсчёта наблюдателя \cite{rafelski2004crawling}. Поэтому, интересующее исследователя движение внутриклеточных структур состоит из движения непосредственно самих структур и движения клетки. Следовательно, нужно установить соответствие между стационарной системой и системой отсчёта подвижной клетки для того, чтобы скомпенсировать движение клетки и анализировать движение внутриклеточных структур относительно стационарной системы отсчета, что в свою очередь требует точный и объективный метод для определения позиции и ориентации клетки в каждом кадре последовательности.

Отследить позицию и ориентацию клетки можно путём нахождения центра клетки \cite{jahne2004practical}, \cite{soo2005large}. Когда клетка не деформируется, и меняется интенсивность внутри клетки, то можно найти поворот и смещение клетки на основе ее формы. Это ограничивает выбор типа клеток и того, каким образом проведено наблюдение. В определённых случаях эти условия нельзя соблюсти из-за особенностей клетки или из-за ограничений эксперимента.
Когда клетка деформируется и интенсивность немного меняется с течением времени, нужен
более сложный подход, который учитывает небольшие изменения интенсивности и деформацию клетки. Суть корреляционного метода, предложенного в \cite{wilson2006correlation}, заключается в следующем: даны два изображения, на которых расположен один и тот же объект, смещенный относительно центра изображения различным образом. Координата максимального значения кросс-корреляции данных изображений соответствует относительному смещению между изображениями \cite{gonzalez2002digital}. Поскольку такой метод основан непосредственно на анализе интенсивности пикселей и совмещении участков клетки с одинаковой (близкой) интенсивностью, то его можно использовать для отслеживания движения клетки, если её форма не существенно изменяется между соседними изображениями.

 Кросс-корреляция --- один из нескольких известных походов к решению задачи совмещения изображений \cite{zitova2003image}, \cite{maes2003medical}. Для совмещения медицинских изображений в работе \cite{hutton2003software} был предложен подход, в котором помимо поиска смещения используется итеративная локальная оптимизация параметрического пространства возможных преобразований. Этот метод хорошо подходит для нежёсткого совмещения изображений одного объекта, полученными из различных типов изображений, например МРТ и КТ.

В данной работе описывается реализация двух-шагового метода, основанного на кросс-корреляции, который оценивает поворот и смещение клетки для двух последовательных кадров набора изображений. Идея метода была предложена в работах \cite{reddy1996fft}, \cite{sheng1986experiments}, однако добавление динамического маскирования и частотной полосовой фильтрации делает возможным успешно применить данный метод к объектам с нежёсткими и динамичными контурами, таким как клетки. Данный двухступенчатый метод работает надёжно для компенсации движения живой клетки, а его производительность превосходит  производительность методов итеративной оптимизации в точности и скорости вычислений.
\chapter{Цель работы}
Целью данной работы является разработка метода компенсации поворота и смещения живой клетки в последовательности изображений.
\chapter{Основная часть}
%\addcontentsline{toc}{chapter}{Основная часть}
\section{Постановка задачи}
Даны два изображения в градациях серого \( I_\text{1} \) и \( I_\text{2} \), на которых показаны клетки, слегка отличающиеся формой, одна из которых повёрнута и смещена относительно другой. Также даны бинарные изображения, представляющие результат сегментации (маски) данных клеток \( M_\text{1} \) и \( M_\text{2} \) соответственно. Требуется найти угол поворота и вектор смещения клетки на изображении \( I_\text{2} \) относительно её положения на изображении \( I_\text{1} \).

\section{Преобразование Фурье}
Пусть \(f(x,y)\) --- заданное изображение. Двумерное \emph{преобразование Фурье} изображения \(f\), обозначим \(F(u,v)\) задается уравнением: \begin{equation} F(u,v) = \int_{-\infty}^{+\infty} \int_{-\infty}^{+\infty} f(x,y)e^{-j2\pi(ux + vy)}\,dx\,dy \end{equation} Обратное преобразование Фурье задается уравнениями:\begin{equation} f(x,y) = \int_{-\infty}^{+\infty} \int_{-\infty}^{+\infty} F(u,v)e^{j2\pi(ux + vy)}\,du\,dv \end{equation} Если \(x\) и \(y\) --- пространственные координаты, то \(u\) и \(v\) --- пространственные частоты вдоль осей \(x\) и \(y\) соответственно. Таким образом, зная \(F(u,v)\), можно восстановить \(f(x,y)\) с помощью обратного преобразования Фурье. Величина \(|F(u,v)|\) называется спектром функции \(F(u,v)\).

\section{Кросс-корреляция}
\emph{Кросс-корреляция} двух изображений \(f(x, y)\) и \(g(x, y)\), обозначим \(f \otimes g\),  задается уравнением: \begin{equation} f \otimes g = \int_{-\infty}^{+\infty}\int_{-\infty}^{+\infty}f(u,v)g(u+x, v+y)\,du\,dv \label{cor0} \end{equation} результатом которого является функция переменных  \(x\) и \(y\). В случае, если кросс-корреляция происходит в одном направлении, то она задается уравнением:  \begin{equation} f \otimes g = \int_{-\infty}^{+\infty}f(x,y)g(u+x,y)\,du \label{c} \end{equation}Кросс-корреляция обладает свойством:  \begin{equation} f(x,y) * g(x,y) = f(-x, -y) \otimes g(x, y)\end{equation} где \(*\) --- операция свёртки. Поэтому по теореме о свертке, которую можно сформулировать так: \[f(x,y) * g(x,y) \Longleftrightarrow G(u,v)F(u,v)\]
и в обратную сторону  \[f(x,y) g(x,y) \Longleftrightarrow G(u,v)*F(u,v)\]
кросс-корреляция может быть эффективно вычислена используя преобразование Фурье соответствующих изображений \(F\) и \(G\) \cite{gonzalez2002digital}, \cite{bracewell1986fourier} : \begin{equation} f  \otimes g = \mathcal{F}^{-1}(\bar{F} G) \label{cor} \end{equation}
где \(\bar F\) - комплексное сопряжение \(F\), а \(\mathcal{F}\)  --- оператор преобразования Фурье. С помощью кросс-корреляции можно найти только смещение. Если объект повернут, то его относительное смещение нельзя найти с помощью двумерной кросс-корреляции. Также шум и другие структуры на изображении, не считая области исследования, портят результат  кросс-корреляции.

\section{Кросс-корреляционный метод совмещения изображений}
На двух данных изображениях расположен объект, который смещён на одном изображении относительно другого. Это смещение можно найти, применив функцию кросс-корреляции к изображениям. Если изображения можно преобразовать к такому виду, в котором поворот объекта на исходных изображениях представляется как смещение, то, применив кросс-корреляцию, можно легко найти соответствующее смещение.
\begin{figure}[th!]
\centering

\caption{a) Исходное изображение \(I_1\); б) Изображение \(I_2\): объект на  \(I_2\) смещен относительно объекта на \(I_1\); в) Полярное преобразование \(P_1\) изображения \(I_1\); г)~Полярное преобразование \(P_2\)  изображения \(I_2\).}
\label{polar}
\end{figure}
Например, поворот вокруг центра изображения можно представить как смещение изображения в полярной системе координат (полярное преобразование изображения) \cite{reddy1996fft}: прямоугольные оси полярного преобразования изображения представляют собой радиус и угол исходного изображения (будем считать, что углу соответствует горизонтальная ось, а радиусу - вертикальная). Поворот вокруг начала координат исходного изображения (область, определяемая нулевым радиусом в полярном преобразовании) становится смещением вдоль оси углов в соответствующем полярном преобразовании. Однако такое представление не инвариантно относительно смещения: смещение объекта в исходном изображении резко меняет полярное преобразование этого изображения (рис.~\ref{polar}). Более того, поворот вокруг центра, не совпадающего с началом изображения --- более сложное преобразование в полярном представлении (рис.~\ref{fft}).
\begin{figure}[h!]
\centering
\caption{Представление поворота через смещение}
\label{fft}
\end{figure}
Следовательно, если не знать центр поворота и смещение между изображениями, то нельзя применять полярное преобразование. Аналогично с помощью кросс-корреляции нельзя найти смещение, если не знать угла поворота объекта на изображении.

Однако, вышеописанные проблемы можно решить с помощью использования преобразования Фурье. Двумерное преобразование Фурье имеет свойства, делающие его очень подходящим для нахождения такого представления изображения, в котором поворот объекта на исходном изображении можно представить как смещение. Во-первых, при повороте изображения вокруг любой точки, поворачивается и спектр преобразования Фурье данного изображения вокруг своего центра \cite{sheng1986experiments}, \cite{bracewell1986fourier}: если \(x = r cos \theta, y = r sin \theta, u = w cos\phi,  v = w sin\phi\), то \(f(x,y)\) и  \(F(u,v)\) при переводе в полярные координаты станут \(f(r,\theta)\) и \(F(w,\phi)\) соответственно. По свойству преобразования Фурье получим:
\begin{equation}  f(r, \theta + \theta_0) \Leftrightarrow F(w, \phi + \theta_0)\end{equation}
Другими словами, поворачивая \(f(x,y)\) на угол \(\theta_0\)   мы поворачиваем и \(F(u,v)\) на тот же угол.
Во-вторых, так как информация о позиции объекта на изображении содержится в фазе преобразования Фурье, то спектр преобразования Фурье инвариантен относительно смещения объекта: если \((a,b)\) --- вектор смещения, то
\begin{equation} \mathcal F\{f(x - a, y - b)\} = F(u, v) e^{-j2\pi(au + bv)} \end{equation}
где \(f(x,y)\) --- исходное изображение, \(F(u,v)\) --- его преобразование Фурье, \(\mathcal{F}\)  ---~оператор преобразования Фурье.
 Таким образом, можно вычислить полярное преобразование спектра преобразования Фурье и с помощью кросс-корреляции определить смещение, которое будет соответствовать искомому углу поворота изображений (рис. \ref{fft}).  Данный подход работает не всегда: если спектр преобразования Фурье симметричный, то угол поворота нельзя найти вышеописанным методом. Однако, в применении к клеткам это маловероятно.
 Подобные методы были описаны раннее как методы совмещения изображений, объекты на которых смещены, повернуты относительно друг друга и сняты в разных масштабах \cite{reddy1996fft}, \cite{sheng1986experiments}, \cite{chen1994symmetric}, \cite{erturk2003translation}, \cite{de1987registration}, \cite{horner1984phase}. Вместо кросс-корреляции в некоторых методах  авторы использовали фазовую корреляцию, потому что она дает более узкий пик, чем кросс-корреляция \cite{sheng1986experiments}, \cite{erturk2003translation}, \cite{de1987registration}, \cite{lee1993principles} для двух искусственных ``идеальных'' изображений. Однако, она не выдает пика для реальных изображений. Поскольку живая клетка со временем значительно меняет свою форму, нельзя отследить её движение, совмещая каждый раз текущее изображение с первым изображением, нужно совмещать  с предыдущим изображением последовательности, применив к нему все вычисленные преобразования. Последовательное совмещение соседних кадров необходимо, так как незначительная ошибка на каждой итерации метода совмещения изображений приводит к значительной общей ошибке. Таким образом, двух-шаговый корреляционный метод, предложенный раннее, не подходит для компенсации движения живой клетки. В данной работе представлены некоторые улучшения кросс-корреляционного метода, а именно применение динамического маскирования и частотной полосовой фильтрации.
\begin{figure}[t]
\centering
\caption{Общая схема корреляционного метода}
\label{diag}
\end{figure}
\section{Алгоритм поиска угла поворота}
Чтобы определить область интереса (а именно найти клетку на изображении), мы применяем соответствующие маски к клеткам  на   \( I_\text{1} \) и \( I_\text{2} \). Маска охватывает всю клетку, но не включает никакие другие объекты и черты изображения. Мы применяем низкочастотный фильтр к маске, используя свёртку маски и гауссиана. Для подавления шума мы применяем медианную фильтрацию к изображениям. Затем мы вычитаем из  \( I_\text{1} \) и \( I_\text{2} \)среднюю интенсивность фона (среднее значение пикселей за пределами маски) в  \( I_\text{1} \) и \( I_\text{2} \) соответственно, и после этого мы умножаем каждое изображение на его маску (рис.~\ref{mask}).
Поскольку преобразование Фурье периодично, разрывы на границах изображения вкладывают свои артефакты в преобразование Фурье. Важно заметить, что спектр преобразования Фурье не содержит таких артефактов, которые могли бы скрыть важные черты представления, используемого для вычисления угла поворота. Также мы хотим, чтобы  \( I_\text{1} \) и  \( I_\text{2} \) были непрерывны на границах, т.е. чтобы значение пикселей плавно доходило до постоянного значения около границы. Благодаря свёртке каждой маски с гауссианом и последующему её применению это требование выполняется.
\begin{figure}[h!]
\begin{minipage}[h]{0.47\linewidth}
\end{minipage}
\hfill
\begin{minipage}[h]{0.47\linewidth}
\end{minipage}
\hfill
\begin{minipage}[h]{0.47\linewidth}
\end{minipage}
\hfill
\begin{minipage}[h]{0.47\linewidth}
\end{minipage}
\caption{a) Исходное изображение \(I_1\); б) Результат медианной фильтрации изображения \(I_1\); в)  Маска клетки, размытая гауссианом; г) Конечный результат.}
\label{mask}
\end{figure}
При вычислении полярного преобразования от спектра преобразования Фурье была использована бикубическая интерполяция, чтобы уменьшить появление выступов между соседними пикселями в представлении полярного преобразования. В предыдущей работе используется логарифмическое полярное преобразование на этом шаге, что позволяет одновременный расчет поворота и масштаба \cite{sheng1986experiments}. Поскольку мы будем применять частотный полосовой фильтр, чтобы использовать для определения угла поворота и смещения только детали определенного размера, мы не можем одновременно вычислить масштаб на этом шаге.
\begin{figure}[h!]
\begin{minipage}[h]{0.49\linewidth}
\end{minipage}
\begin{minipage}[h]{0.49\linewidth}
\end{minipage}
\caption{Слева - полярные преобразования спектра \( I_\text{1} \) (сверху) и  \( I_\text{2} \) (снизу), справа - результат кросс-корреляции (смещение = 10.15 градусов) }
\end{figure}

Чтобы вычислить угол поворота клетки, мы используем полосовой фильтр для выбора пространственных частот, соответствующих масштабу объектов, которые определяют поворот. Учитывая, что вертикальная ось полярного преобразования спектра соответствует частотам, мы можем выполнить фильтрацию, взяв горизонтальную полосу этого представления изображения. Положение и ширина частотной полосы устанавливается экспериментальным путем. Кроме того, так как мы будем использовать только фрагмент полярного преобразования, то данный фрагмент будет единственной частью полярного преобразования, которую нужно посчитать. Это существенно ускоряет метод, поскольку полярное преобразование является наиболее ресурсоёмким шагом в реализации.
Также мы вычисляем только левую половину этой полосы (углы от 0 до \(\pi\) ), поскольку область \([\pi,2\pi]\) периодически повторяет область \([0,\pi]\) из-за сопряженной симметрии преобразования Фурье изображения \cite{sheng1986experiments}. Это означает, что после вычисления угла поворота \(\theta\),  смещение должно быть вычислено для изображения, повернутого на \(\theta\) и на \(\theta + \pi\) (кросс-корреляция с наибольшим максимальным значением соответствует фактическому углу поворота). Если, с другой стороны, можно предположить, что угол поворота лежит в определённых пределах, то не нужно проверять оба значения. Для наших данных предположим, что угол лежит от \(0\) до \(\pi\).

Последним шагом в вычислении угла поворота является кросс-корреляция. Заметим, что при вычислении угла поворота может быть выполнена одномерная кросс-корреляция, поскольку нам нужна только координата максимального значения по оси, соответствующей углу в полярном преобразовании. В представленных раннее алгоритмах совмещения изображений, фазовая корреляция давала более узкий пик, чем корреляция, вычисленная как в работах \cite{sheng1986experiments}, \cite{erturk2003translation}, \cite{de1987registration}, \cite{lee1993principles}. Метод фазовой корреляции предполагает, что изображения \(f\) и \(g\) отличаются между собой только смещением. Однако, наши отфильтрованные  полярно-преобразованные спектры изображений отличаются значительно больше, чем просто смещением. Было обнаружено, что фазовая корреляция этих представлений не  имеет пика, соответствующего относительному смещению между изображениями. Для наших изображений клеток, похожих между собой, но не идентичных, требуется стандартная кросс-корреляция.
Кросс-корреляция вычисляется с помощью преобразования Фурье (\ref{cor}):
\[ M'_i = M_i * G, i = 1,2\] \[I_i' = I_i * MF, i = 1,2\]
\[I_i'' = I_i' \cdot M_i', i = 1,2\] \[ F_i = \mathcal F(I_i''), i = 1,2\] \[P_i = \mathcal P(F_i), i = 1,2\]
\[P_i' = P_i \cdot BF, i = 1,2\]
\[Res = P_1' \otimes P_2', i = 1,2\]
где \(I_1\) и \(I_2\) -- исходные изображения, \(M_i\) -- маска изображения \(I_i\), \(G\) -- гауссиан, \(MF\) -- медианный фильтр, \(F_i\) -- преобразование Фурье изображения \(I_i\), \(\mathcal F\) -- оператор преобразования Фурье, \(P_i\) -- полярное преобразование \(F_i\), \(\mathcal P\) -- оператор полярного преобразования, \(BF\) --частотный полосовой фильтр, \(Res\) -- результат кросс-корреляции, , \(i = 1,2\).

\section{Алгоритм поиска вектора смещения}
Определив угол вращения клетки между изображениями и повернув \(I_2\) так, чтобы клетка  имела такое же положение как и в \(I_1\), получим новое изображение \(I_2'\). Тогда можно использовать кросс-корреляцию для расчета смещения клетки между \(I_1\) и \(I_2'\).
Однако смещение зависит от того, вокруг какой точки изображение \(I_2\) было повёрнуто. В данной реализации точка поворота была определена как центр масс клетки. Также точку поворота клетки можно определить как центр масс вершин контуров клетки или с помощью маркеров объема для достижения более точного результата.
Кросс-корреляция, рассчитывается таким же образом, что и в алгоритме поиска угла поворота.
\chapter{Экспериментальные результаты}
Разработанный метод был применён к последовательности изображений живой клетки из 33 кадров, которые были сделаны в разные моменты времени. Алгоритм применялся для каждой последовательной пары кадров, результаты каждой следующей итерации суммировались со всеми предыдущими. Результаты поиска угла поворота и смещения относительно первого кадра приведены в таблице \ref{rtable}.
На рисунке \ref{res}  проиллюстрированы результаты работы метода и показаны изображения, повернутые таким образом, что клетка на каждом кадре расположена как на первом кадре последовательности (т.е. как в начальный момент времени).
\begin{figure}[t]
\centering
\caption{Результаты метода: сверху расположена исходная последовательность изображений движения клетки в различные моменты времени; снизу расположены результаты работы метода для каждого изображения соответственно.}
\label{res}
\end{figure}

 \begin{table}[h]
\centering
\begin{tabular}{| p{0.11\textwidth} | p{0.39\textwidth} |  p{0.39\textwidth}|}
  \hline \emph{Время, c} & \emph{Угол поворота относительно первого изображения последовательности} & \emph{Величина вектора смещения относительно первого изображения последовательности} \\ \hline
  t = 3 & 11.5038 & 19.3132  \\ \hline
  t = 5 & 16.9173 & 12.1655 \\ \hline
  t = 7 & 32.4812 & 28.2312 \\ \hline
  t = 9 & 39.9248 & 18.1453  \\ \hline
\end{tabular}
\caption{ Результаты метода}
\label{rtable}
\end{table}

\chapter{Заключение}
В данной работе был проведён обзор существующих подходов к решению задачи предсказания цен акций компаний,
основывающихся на тональности новостных и других источников информации для инвесторов.
Был проведён эксперимент по выявлению трендов на рынке(???).
Полученные результаты говорят о низкой результативности подхода наивного подхода.
Были высказаны предложения по улучшению алгоритма и обозначены возможные направления дальнейшей работы.

\addcontentsline{toc}{chapter}{Список литературы} % добавляем пункт "Литература" в оглавление
\renewcommand{\bibname}{Список литературы}
\bibliographystyle{unsrt}
\bibliography{bibl}

\end{document}
